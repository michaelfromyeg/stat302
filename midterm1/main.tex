\documentclass{article}

%%%%%%%%%%%%%%%%%%%%%%%%%%%%%%%%%%%%%%%%%
% Lachaise Assignment
% Structure Specification File
% Version 1.0 (26/6/2018)
%
% This template originates from:
% http://www.LaTeXTemplates.com
%
% Authors:
% Marion Lachaise & François Févotte
% Vel (vel@LaTeXTemplates.com)
%
% License:
% CC BY-NC-SA 3.0 (http://creativecommons.org/licenses/by-nc-sa/3.0/)
% 
%%%%%%%%%%%%%%%%%%%%%%%%%%%%%%%%%%%%%%%%%

%----------------------------------------------------------------------------------------
%	PACKAGES AND OTHER DOCUMENT CONFIGURATIONS
%----------------------------------------------------------------------------------------

\usepackage{amsmath,amsfonts,stmaryrd,amssymb} % Math packages

\usepackage{enumerate} % Custom item numbers for enumerations

\usepackage[ruled]{algorithm2e} % Algorithms

\usepackage[framemethod=tikz]{mdframed} % Allows defining custom boxed/framed environments

\usepackage{listings} % File listings, with syntax highlighting
\lstset{
	basicstyle=\ttfamily, % Typeset listings in monospace font
}

%----------------------------------------------------------------------------------------
%	DOCUMENT MARGINS
%----------------------------------------------------------------------------------------

\usepackage{geometry} % Required for adjusting page dimensions and margins

\geometry{
	paper=a4paper, % Paper size, change to letterpaper for US letter size
	top=2.5cm, % Top margin
	bottom=3cm, % Bottom margin
	left=2.5cm, % Left margin
	right=2.5cm, % Right margin
	headheight=14pt, % Header height
	footskip=1.5cm, % Space from the bottom margin to the baseline of the footer
	headsep=1.2cm, % Space from the top margin to the baseline of the header
	%showframe, % Uncomment to show how the type block is set on the page
}

%----------------------------------------------------------------------------------------
%	FONTS
%----------------------------------------------------------------------------------------

\usepackage[utf8]{inputenc} % Required for inputting international characters
\usepackage[T1]{fontenc} % Output font encoding for international characters

\usepackage{XCharter} % Use the XCharter fonts

%----------------------------------------------------------------------------------------
%	COMMAND LINE ENVIRONMENT
%----------------------------------------------------------------------------------------

% Usage:
% \begin{commandline}
%	\begin{verbatim}
%		$ ls
%		
%		Applications	Desktop	...
%	\end{verbatim}
% \end{commandline}

\mdfdefinestyle{commandline}{
	leftmargin=10pt,
	rightmargin=10pt,
	innerleftmargin=15pt,
	middlelinecolor=black!50!white,
	middlelinewidth=2pt,
	frametitlerule=false,
	backgroundcolor=black!5!white,
	frametitle={Command Line},
	frametitlefont={\normalfont\sffamily\color{white}\hspace{-1em}},
	frametitlebackgroundcolor=black!50!white,
	nobreak,
}

% Define a custom environment for command-line snapshots
\newenvironment{commandline}{
	\medskip
	\begin{mdframed}[style=commandline]
}{
	\end{mdframed}
	\medskip
}

%----------------------------------------------------------------------------------------
%	FILE CONTENTS ENVIRONMENT
%----------------------------------------------------------------------------------------

% Usage:
% \begin{file}[optional filename, defaults to "File"]
%	File contents, for example, with a listings environment
% \end{file}

\mdfdefinestyle{file}{
	innertopmargin=1.6\baselineskip,
	innerbottommargin=0.8\baselineskip,
	topline=false, bottomline=false,
	leftline=false, rightline=false,
	leftmargin=2cm,
	rightmargin=2cm,
	singleextra={%
		\draw[fill=black!10!white](P)++(0,-1.2em)rectangle(P-|O);
		\node[anchor=north west]
		at(P-|O){\ttfamily\mdfilename};
		%
		\def\l{3em}
		\draw(O-|P)++(-\l,0)--++(\l,\l)--(P)--(P-|O)--(O)--cycle;
		\draw(O-|P)++(-\l,0)--++(0,\l)--++(\l,0);
	},
	nobreak,
}

% Define a custom environment for file contents
\newenvironment{file}[1][File]{ % Set the default filename to "File"
	\medskip
	\newcommand{\mdfilename}{#1}
	\begin{mdframed}[style=file]
}{
	\end{mdframed}
	\medskip
}

%----------------------------------------------------------------------------------------
%	NUMBERED QUESTIONS ENVIRONMENT
%----------------------------------------------------------------------------------------

% Usage:
% \begin{question}[optional title]
%	Question contents
% \end{question}

\mdfdefinestyle{question}{
	innertopmargin=1.2\baselineskip,
	innerbottommargin=0.8\baselineskip,
	roundcorner=5pt,
	nobreak,
	singleextra={%
		\draw(P-|O)node[xshift=1em,anchor=west,fill=white,draw,rounded corners=5pt]{%
		Question \theQuestion\questionTitle};
	},
}

\newcounter{Question} % Stores the current question number that gets iterated with each new question

% Define a custom environment for numbered questions
\newenvironment{question}[1][\unskip]{
	\bigskip
	\stepcounter{Question}
	\newcommand{\questionTitle}{~#1}
	\begin{mdframed}[style=question]
}{
	\end{mdframed}
	\medskip
}

%----------------------------------------------------------------------------------------
%	WARNING TEXT ENVIRONMENT
%----------------------------------------------------------------------------------------

% Usage:
% \begin{warn}[optional title, defaults to "Warning:"]
%	Contents
% \end{warn}

\mdfdefinestyle{warning}{
	topline=false, bottomline=false,
	leftline=false, rightline=false,
	nobreak,
	singleextra={%
		\draw(P-|O)++(-0.5em,0)node(tmp1){};
		\draw(P-|O)++(0.5em,0)node(tmp2){};
		\fill[black,rotate around={45:(P-|O)}](tmp1)rectangle(tmp2);
		\node at(P-|O){\color{white}\scriptsize\bf !};
		\draw[very thick](P-|O)++(0,-1em)--(O);%--(O-|P);
	}
}

% Define a custom environment for warning text
\newenvironment{warn}[1][Warning:]{ % Set the default warning to "Warning:"
	\medskip
	\begin{mdframed}[style=warning]
		\noindent{\textbf{#1}}
}{
	\end{mdframed}
}

%----------------------------------------------------------------------------------------
%	INFORMATION ENVIRONMENT
%----------------------------------------------------------------------------------------

% Usage:
% \begin{info}[optional title, defaults to "Info:"]
% 	contents
% 	\end{info}

\mdfdefinestyle{info}{%
	topline=false, bottomline=false,
	leftline=false, rightline=false,
	nobreak,
	singleextra={%
		\fill[black](P-|O)circle[radius=0.4em];
		\node at(P-|O){\color{white}\scriptsize\bf i};
		\draw[very thick](P-|O)++(0,-0.8em)--(O);%--(O-|P);
	}
}

% Define a custom environment for information
\newenvironment{info}[1][Info:]{ % Set the default title to "Info:"
	\medskip
	\begin{mdframed}[style=info]
		\noindent{\textbf{#1}}
}{
	\end{mdframed}
}


\title{STAT302: Midterm \#1 Revision}

\author{Michael DeMarco\\ \texttt{mdemar01@student.ubc.ca}}

\date{University of British Columbia --- October 12th, 2021}

\begin{document}

\maketitle % Print the title

\section{Number of possible pins}

I got both parts of this problem wrong due to some mis-steps while approaching them using permutations (and eliminating duplicates) instead of combinations.

\subsection{Two 0s in the pin}

The solution is $\binom{4}{2} \times 9^2 = 486$. The solution given is for 5-spaces in the pin, but the solution is generally $\binom{n}{2} \times 9^{n-2}$ where $n$ is the number of spaces.

The solution I gave on WebWork was as follows

$$\frac{9\times4!}{2!\times2!} + \frac{9\times8\times4!}{2!} = 918$$

I was (overly) concerned about the ordering of the digits, so I opted to use permutations, and wanted to ensure I eliminated duplicates. There were two cases here, a string like (a) "00XX" and a string like (b) "00XY", leading to the two terms. We always have $4!$ ways of arranging 4 characters. In (a), we eliminate two sets of duplicates and only have 9 choices for X (and only 1 choice for the other X). Likewise in (b), we only have to cancel one set of duplicates and have $9 \times 8$ choices for X and Y.

It becomes clear how "close" (or far) I was if we just re-write my solution in terms of $\binom{n}{k}$.

\begin{equation}
	\begin{split}
		\frac{1\times1\times9\times1\times4!}{2!\times2!} + \frac{1\times1\times9\times8\times4!}{2!} & = \frac{9\times8\times4!}{2!\times2!} + \frac{9\times8\times4!\times2!}{2!\times2!} \\
		& = \frac{9\times8\times4! + 9\times8\times4!\times2!}{2!\times2!} \\
		& = \frac{9\times4!\times(1 + 8\times2!)}{2!\times2!} \\
		& = \frac{9\times4!\times17}{2!\times2!} \\
		& = \frac{4!}{2!\times2!}\times9\times17 \\
		& = \binom{4}{2}\times9\times17\\
		& = 918
	\end{split}
\end{equation}

I considered solutions like $\frac{9\times9\times4!}{2!2!}$ but the magic extra $2!$ on the bottom for "some other duplicate" (other than "00") made me believe it was incorrect so I scratched it out. The proposed solution I gave was, after some squinting, actually correct everywhere in except one term. I was aware that I need to order 4 things, to eliminate duplicate in the zeroes, and to eliminate "some other duplicate" in the other letters, but overly complicated my approach. To correctly do this, I should have added a second $2!$ to the bottom of the second term to account for duplicates there as well. Then, it would've been completely correct, as

$$\frac{9\times4!}{2!\times2!} + \frac{9\times8\times4!}{2!\times2!}=54+432=486$$

as desired.

\subsection{1 and 2 in the pin}

(The argument here is practically identical, so feel free to skim a bit.) The solution is $2\binom{4}{2} \times 8^2 = 768$. The solution given is for 5-spaces in the pin, but the solution is generally $2\binom{n}{2} \times 8^{n-2}$ where $n$ is the number of spaces.

The solution I gave on WebWork was as follows

$$\frac{1\times1\times8\times1\times4!}{2!} + \frac{1\times1\times8\times7\times4!}{1} = 1440$$

I was (overly) concerned about the order of the digits and wanted ensure I was handling cases where there were duplicates in the string. In the first case, I imagined we had a string like $\text{"12XX"}$ where X was some digit other than 0 (and the "X"s are of course equal). There are 4! ways to re-arrange these characters and we need to divide by $2!$ to account for re-arranging XX. In the second case, I imagined we had a string like $\text{"12XY"}$. This explains the $8\times7$ part of the term, since we have $8$ choices for X and then just $7$ for Y. The same logic follows from before, but this time we have no duplicates (I assumed).

It becomes clear how "close" I was if we just re-write my term in terms of $\binom{n}{k}$.

\begin{equation}
	\begin{split}
		\frac{8\times4!}{2!} + \frac{8\times7\times4!}{1} & = \frac{8\times4!\times2!}{2!\times2!} + \frac{8\times7\times4!\times2!\times2!}{2!\times2!} \\
		& = \frac{8\times4!\times2! + 8\times7\times4!\times2!\times2!}{2!\times2!} \\
		& = \frac{8\times4!\times2!\times(1 + 7\times2!)}{2!\times2!} \\
		& = \frac{8\times4!\times2!\times15}{2!\times2!} \\
		& = \frac{4!}{2!\times2!} \times 8\times15\times2 \\
		& = \binom{4}{2} \times 8\times15\times2 \\
		& = 1440
	\end{split}
\end{equation}

I considered solutions like $\frac{8\times8\times4!}{2!2!}$, thinking the second $2!$ could somehow magically erase all the "XX" cases above (which is correct) but scratched them out as I was concerned about eliminating too many cases... alas. The proposed solution I gave was (after some squinting) correct everywhere except for one term, due to over-counting certain cases. Similar to the first part, the term again missed was a $2!$, and again in this case, it should be placed in the denominator in the second term. We'd see

$$\frac{8\times4!}{2!} + \frac{8\times7\times4!}{2!} = 96 + 672 = 768$$

as desired.

\vspace{3mm}
\textbf{The moral?} In both questions, I tried to "split" the question into two cases, when you can't neatly do that in these counting questions and thus \textit{underestimated} the number of duplicates I had. A missing $2!$ in both parts would've earned me full points.

(I've summarized my work shown here, but I'm happy to walk through the submission I uploaded to Canvas as well and "prove" that this was my thinking via the scratch-work. Apologies for the long write-up, but this one took me a minute to see!)

\end{document}
